\documentclass{beamer}
\usetheme{metropolis}           % Use metropolis theme
\metroset{numbering=fraction}
\usepackage[T2A]{fontenc}
\usepackage[utf8]{inputenc}
\usepackage[english,russian]{babel}
\title{Декларативное программирование}
\subtitle{Семинар №1, группа 22215}
\author{Завьялов А.А.}
\date{5 сентября 2022 г.}
\institute{Кафедра систем информатики ФИТ НГУ}
\begin{document}
  \maketitle
  \addtocounter{framenumber}{1}
  % \section{First Section}
  \begin{frame}{Семинарист}
    \begin{block}{Завьялов Антон (Алексеевич)}
        \begin{itemize}
            \item a.zavyalov@g.nsu.ru
            \item \includegraphics[height=\fontcharht\font`\B]{media/telegram128} \url{https://t.me/arx_dukalis}
            \item + 7 (960) 956 61 14
        \end{itemize}
    \end{block}
  \end{frame}
  \begin{frame}{Группа в Telegram}
  \includegraphics[width=4cm]{media/telegram-invite-22215}
  
  \includegraphics[height=\fontcharht\font`\B]{media/telegram128} \url{https://bit.ly/haskell22215}
  \end{frame}
  \begin{frame}{План на первые 2 семинара}
     \begin{block}{Знакомство с языком программирования Haskell}
          \begin{itemize}
              \item Установка и настройка инструментария
              \item Введение в синтаксис и базовые типы данных
          \end{itemize}
      \end{block}
  \end{frame}
  %\begin{frame}{Пока мы не начали...}
  %    \begin{block}{Что такое программирование?}
  %         \pause Процесс построения компьютерных программ
  %         
  %         \pause Программы можно конструировать разными способам
  %    \end{block}
  %\end{frame}
  \begin{frame}{Вводные данные о Haskell}
      \begin{block}{Немного про язык}
        \begin{itemize}
            \item Больше академический, чем промышленный
                \begin{itemize}
                    \item девиз -- ''avoid \textbf{success at all costs}''\footnote{пер. -- ''избегать успеха любой ценой''}
                    \item или всё-таки ''\textbf{avoid success} at all costs''?
                \end{itemize}
            \item Назван в честь математика и логика Хаскелла Карри
            \item Математическая база -- типизированное $\lambda$-исчисление 2-го порядка
            \item Стандартизованный\footnote{см. \alert{\href{Haskell2010}{https://www.haskell.org/onlinereport/haskell2010/}}}, одна официальная реализация -- Glasgow Haskell Compiler
            \item Компилируемый
            \item Интерпретируемый
        \end{itemize}
      \end{block}
  \end{frame}
  \begin{frame}{Установка и использование Haskell}
      \begin{block}{Набор инструментов Haskell состоит из}
\begin{itemize}
    \item Glasgow Haskell Compiler (GHC) -- компилятор
    \item Cabal -- пакетный менеджер и система сборки
    \item Stack -- (еще один) пакетный менеджер и система сборки
\end{itemize}
      \end{block}
      \begin{block}{Рекомендуемый способ установки тулчейна -- GHCup}
        
        \center\alert{{\url{https://www.haskell.org/ghcup/}}}
      \end{block}
  \end{frame}
  \begin{frame}{Установка и использование Haskell -- редакторы}
      \begin{itemize}
          \item Visual Studio Code
            \begin{itemize}
                \item есть хороший \alert{\href{https://marketplace.visualstudio.com/items?itemName=haskell.haskell}{плагин}}
            \end{itemize}
            \item Emacs
            \item (Neo)Vim
            \item ...
            \item \href{https:\\leksah.org}{Leksah}
      \end{itemize}
      \begin{itemize}
          \item \url{http://repl.it/}
            \begin{itemize}
                \item онлайн-редактор
            \end{itemize}
      \end{itemize}
  \end{frame}
  \begin{frame}{Glasgow Haskell Compiler -- команды терминала}
      \begin{itemize}
          \item \texttt{ghc} -- компилирует исходный код в объектный/исполняемый файл
          \item \texttt{runghc}, \texttt{runhaskell} -- выполняет переданный код
          \item \texttt{ghci} -- запускает сессию \texttt{REPL}, интерактивное окружение для разработки
      \end{itemize}
  \end{frame}
  \begin{frame}{GHCi -- REPL\footnote{REPL (от англ. ''\alert{r}ead-\alert{e}val-\alert{p}rint \alert{l}oop'') -- простейшая организация интерактивной среды программирования в рамках интерфейса командной строки, выраженная в повторении последовательности чтения пользовательского кода, его исполнения и печати результатов в поток вывода.} для Haskell}
      \begin{block}{Что может делать GHCi}
        \begin{itemize}
            \item вычислять выражения
                \begin{itemize}
                    \item печатать информацию об их типах
                \end{itemize}
            \item выполнять программы в режиме интерпретации
            \item загружать модули, \textit{скомпилированные} GHC в машинный код
        \end{itemize}
      \end{block}
  \end{frame}
  \begin{frame}{Hackage и Hoogle}
      \begin{block}{Hackage (\url{hackage.haskell.org})}
        \begin{itemize}
            \item Хранилище пакетов Haskell (Как \texttt{PyPI} для \texttt{Python})
            \item Можно, например, изучить документацию к \alert{\href{https://hackage.haskell.org/package/base}{базовому набору библиотек}}
        \end{itemize}
      \end{block}
      \begin{block}{Hoogle (\url{hoogle.haskell.org})}
        \begin{itemize}
            \item Поисковый движок для кода на Haskell
        \end{itemize}
      \end{block}
  \end{frame}
    \begin{frame}{Вводные данные о Haskell}
      %\begin{itemize}
      %    \item Назван в честь Хаскелла Карри, американского математика и логика
      %    \item Первый релиз -- 1990 г.
      %    \item Последний стандарт -- Haskell2010
      %    \item Основная реализация -- Glasgow Haskell Compiler
      %\end{itemize}
      
      \begin{block}{Язык}
          \begin{itemize}
              \item Функциональный
                \begin{itemize}
                    \item любая программа представлена комбинацией \textit{функций} (в математическом смысле)
                \end{itemize}
              \item Чистый
                \begin{itemize}
                    \item функции \textbf{детерминированные}\footnote{для заданных входных данных функция всегда возвращает одинаковый результат}
                    \item функции не имеют \textbf{побочных эффектов}\footnote{не ссылаются на данные, значения которых могут измениться в процессе работы программы и не производит таких изменений}
                \end{itemize}
              \item Ленивый (по умолчанию)
                \begin{itemize}
                    \item \textit{aka} \textit{нестрогие вычисления}, \textit{отложенные вычисления}
                    \item Вычисления производятся только по необходимости
                \end{itemize}
          \end{itemize}
      \end{block}
  \end{frame}
  \begin{frame}{Вводные данные о Haskell}
      \begin{block}{Типизация}
          \begin{itemize}
              \item Статическая
                \begin{itemize}
                    \item типы проверяются до исполнения программы
                \end{itemize}
              \item Сильная
              \item Неявная
                \begin{itemize}
                    \item вывод типов
                \end{itemize}
          \end{itemize}
      \end{block}
      \begin{block}{Полиморфизм}
        \begin{itemize}
            \item параметрический
            \item ad-hoc
        \end{itemize}
      \end{block}
  \end{frame}
  \begin{frame}{Что почитать?}
      \begin{itemize}
          \item Изучай Haskell во имя добра! -- Миран Липовача
          \begin{itemize}
              \item Бесплатно, на английском: \url{http://learnyouahaskell.com/}
          \end{itemize}
          \item Программируй на Haskell -- Уилл Курт
      \end{itemize}
      \begin{itemize}
          \item \alert{\href{https://hbr.github.io/Lambda-Calculus/lambda2/lambda.html}{Programming with Lambda Calculus}} -- Helmut Brandl
      \end{itemize}
  \end{frame}
  \section{Q\&A}
\end{document}