\documentclass{beamer}
\usetheme{metropolis}           % Use metropolis theme
\metroset{numbering=fraction}
\usepackage[T2A]{fontenc}
\usepackage[utf8]{inputenc}
\usepackage[english,russian]{babel}
\usepackage{minted}
\usepackage{soul}

\newcommand{\hs}[1]{\mintinline{haskell}{#1}}
\newcommand{\bhs}[1]{\fbox{\mintinline{haskell}{#1}}}

\title{Декларативное программирование}
\subtitle{Семинар №10, группа 22215}
\author{Завьялов А.А.}
\date{7 ноября 2022 г.}
\institute{Кафедра систем информатики ФИТ НГУ}
\begin{document}
  \maketitle
  \begin{frame}[fragile]{Можно ли перечислить пару перечислимых значений?}
      \begin{minted}{haskell}
instance (Enum a, Enum b) => Enum (a, b) where
    toEnum n = (toEnum n, toEnum n)
    fromEnum (a, b) = ???
      \end{minted}
      \begin{block}{Ожидаемое поведение}
      \begin{minted}{haskell}
GHCI> [(1,2)..(3,4)]
[(1,2),(2,3),(3,4)]
      \end{minted}
      \end{block}
  \end{frame}
  \section{Знакомство с монадами}
  \begin{frame}[fragile]{Знакомые нам виды вычислений}
      \begin{itemize}
          \item Применение функции к аргументу (аргументам):
          \begin{itemize}
              \item \hs{f :: a -> b, x :: a} $\Rightarrow$ \hs{f $ x :: b}
              \item \hs{g :: a -> b -> c, x :: a, y :: b} $\Rightarrow$ \hs{g $ x y :: c}
          \end{itemize}
          \item Применение функции к аргументу в контексте:
          \begin{itemize}
              \item \hs{f :: a -> b, x ::} \bhs{a} $\Rightarrow$ \hs{f <$> x ::} \bhs{b}
              \item \hs{f :: a -> b} $\Rightarrow$ \hs{fmap f ::} \bhs{a} \hs{->} \bhs{b}
          \end{itemize}
          \item Применение функции \textit{в контексте} к аргументу в контексте:
          \begin{itemize}
              \item \hs{g ::} \bhs{a -> b}, \hs{x ::} \bhs{a} $\Rightarrow$ \hs{g <*> x ::} \bhs{b}
              \item Применение бинарной операции к операндам в контексте:
                  \hs{h :: a -> b -> c, x ::} \bhs{a}, \hs{y :: } \bhs{b} \Rightarrow \hs{liftA2 h x y ::} \bhs{c}
          \end{itemize}
      \end{itemize}
  \end{frame}
  \begin{frame}[fragile]{Незнакомые нам виды вычислений}
      \begin{itemize}
          \item \hs{k :: a ->} \bhs{b}, \hs{x ::} \bhs{a} \Rightarrow \hs{k}~ \alert{\texttt{???}}~\hs{x ::}~\bhs{b}
      \end{itemize}
      \begin{block}{Пример}
      \begin{minted}{haskell}
safeHead :: [a] -> Maybe a
safeHead [] = Nothing
safeHead (x:_) = Just x

safeTail :: [a] -> Maybe [a]
safeTail [] = Nothing
safeTail (_:xs) = Just xs

safeSecond :: [a] -> Maybe a
safeSecond = ??? safeTail ??? safeHead
      \end{minted}
      \end{block}
  \end{frame}
  \begin{frame}[fragile]{Класс типов \texttt{Monad}}
      \begin{minted}{haskell}
class Applicative m => Monad m where
    {-# MINIMAL (>>=) #-}
    (>>=)       :: m a -> (a -> m b) -> m b
    (>>)        :: m a -> m b -> m b
    return      :: a -> m a
infixl 1 >>, >>=
    
>safeSecond xs = safeTail xs >>= safeHead
>safeSecond [1] = Nothing
>safeSecond [1,2] = Just 2
      \end{minted}
      \begin{itemize}
          \item Функции типа \hs{a -> m a} называются \textit{стрелками Клейсли}
          \item \hs{(>>=)} (читается ''bind'') называется \textit{оператором монадического связывания}
      \end{itemize}
  \end{frame}
  \begin{frame}[fragile]{Функция \texttt{return :: a -> m a}}
      \begin{itemize}
          \item Определяет тривиальную стрелку Клейсли
          \item \hs{return = pure}
      \end{itemize}
      \begin{minted}{haskell}
toKleisli :: Monad m => (a -> b) -> (a -> m b)
toKleisli f = \x -> return (f x)

>:t toKleisli (*2)
toKleisli (*2) :: (Monad m, Num a) => a -> m a
>(toKleisli (*2) 2) :: Maybe Int
Just 4
>(toKleisli (*2) 2) :: [Int]
[4]
>return 42 :: [Int]
[42]
      \end{minted}
  \end{frame}
  \begin{frame}[fragile]{Оператор \texttt{(>{}>) :: m a -> m b -> m b}}
      \begin{itemize}
          \item Определён по умолчанию:
           
           \hs{m >> k = m >>= \_ -> k}
           \item Если \hs{m} породило значение, отбрасывает \hs{m}, возвращает \hs{k}
      \end{itemize}
      \metroset{block=fill}
      \begin{block}{Пример}
      \begin{minted}{haskell}
>Just 1 >> Just 2
Just 2
>Nothing >> Just 2
Nothning
>[1] >> [2]
[2]
>[] >> [2]
[]
      \end{minted}
      \end{block}
  \end{frame}
  \begin{frame}[fragile]{Другие полезные функции для работы с монадами}
      \begin{itemize}
          \item \hs{(>=>) :: Monad m => (a -> m b) -> (b -> m c)} \hs{-> a -> m c},
           композиция стрелок Клейсли
           \item \hs{(<=<) :: Monad m => (b -> m c) -> (a -> m b)} \hs{-> a -> m c}
           \item \hs{(=<<) :: Monad m => (a -> m b) -> m a -> m b}
            
           \hs{(=<<) = flip (>>=)}
      \end{itemize}
      \metroset{block=fill}
      \begin{block}{Пример}
     \begin{minted}{haskell}
 safeSecond = safeTail >=> safeHead
     \end{minted}
     \end{block}
  \end{frame}
  \begin{frame}{Законы класса типов \texttt{Monad}\footnote{англ. -- monadic laws}}
      Для любого представителя \texttt{Monad} должны выполняться следующие свойства:
      \begin{itemize}
          \item \hs{return a >>= k = k a}
          \item \hs{m >>= return = m}
          \item \textbf{Ассоциативность} \hs{m >>= (\x -> k x >>= h) = (m >>= k) >>= h}
      \end{itemize}
  \end{frame}
  \begin{frame}[fragile]{Законы класса типов \texttt{Monad}}\metroset{block=fill}
      \begin{block}{Композиция Клейсли}
      \begin{minted}{haskell}
infixr 1 >=>
(>=>) :: Monad m => (a -> m b)->(b -> m c)->(a -> m c)
f >=> g =  \x -> f x >>= g
      \end{minted}
      \end{block}
      Выразим свойства через композицию стрелок:
      \begin{itemize}
          \item \hs{return >=> h = h}
          \item \hs{f >=> return = f}
          \item \textbf{Ассоциативность}
          
          \hs{(f >=> g) >=> h = f >= (g >= h)}
      \end{itemize}
  \end{frame}
  \begin{frame}{\texttt{do}-нотация}\metroset{block=fill}
      Синтаксический сахар для упрощенной записи цепочек \textit{монадических вычислений}
\begin{columns}
\begin{column}{0.5\textwidth}
\begin{block}{do-нотация}
\hs{do { e1; e2 }} 
\end{block}
\begin{block}{do-нотация}
\hs{do { p <- e1; e2 }}
\end{block}
\begin{block}{do-нотация}
\hs{do { let v = e1; e2 }} 
\end{block}
\end{column}
\begin{column}{0.5\textwidth}
\begin{block}{Обычный синтаксис}
\hs{e1 >> e2}
\end{block}
\begin{block}{Обычный синтаксис}
\hs{e1 >>= \p -> e2}
\end{block}
\begin{block}{Обычный синтаксис}
\hs{let v = e1 in do e2}
\end{block}
\end{column}
\end{columns}
  \end{frame}
  \begin{frame}[fragile]{Пример использования \texttt{do}-нотации}
\begin{columns}
\begin{column}{0.475\textwidth}
  \begin{minted}{haskell}
safeSecond xs = do
  x <- safeTail xs
  y <- safeHead x
  return y
\end{minted}
\end{column}
\begin{column}{0.05\textwidth}
  $\equiv$
\end{column}
\begin{column}{0.475\textwidth}
  \begin{minted}{haskell}
safeSecond xs = 
  safeTail xs >>= (\x ->
  safeHead x >>= (\y ->
  return y))
\end{minted}
\end{column}
\end{columns}
  \end{frame}
\begin{frame}{Домашнее задание}
    \begin{itemize}
        \item Реализовать функции \hs{fmap, pure, (<*>)} с помощью \hs{return, (>>=)}
    \end{itemize}
\end{frame}
  \section{Q\&A}
\end{document}