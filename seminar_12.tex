\documentclass{beamer}
\usetheme{metropolis}           % Use metropolis theme
\metroset{numbering=fraction}
\usepackage[T2A]{fontenc}
\usepackage[utf8]{inputenc}
\usepackage[english,russian]{babel}
\usepackage{minted}
\usepackage{soul}

\newcommand{\hs}[1]{\mintinline{haskell}{#1}}
\newcommand{\bhs}[1]{\fbox{\mintinline{haskell}{#1}}}

\title{Декларативное программирование}
\subtitle{Семинар №12, группа 22215}
\author{Завьялов А.А.}
\date{21 ноября 2022 г.}
\institute{Кафедра систем информатики ФИТ НГУ}
\begin{document}
  \maketitle
  \begin{frame}[fragile]{Глубокий смысл монадического связывания}
    \begin{minted}{haskell}
class Applicative m => Monad m where
    -- `bind`, оператор монадического связывания
    (>>=)       :: m a -> (a -> m b) -> m b
    (>>)        :: m a -> m b -> m b
    return      :: a -> m a
      \end{minted}
      \begin{block}{Монадические законы}
        \begin{itemize}
          \item \hs{return a >>= k} $\equiv$ \hs{k a}
          \item \hs{m >>= return} $\equiv$ \hs{m}
          \item \textbf{Ассоциативность}\linebreak
                \hs{m >>= (\x -> k x >>= h)} $\equiv$ \hs{(m >>= k) >>= h}
      \end{itemize}
      \end{block}
  \end{frame}
  \begin{frame}[fragile]{Глубокий смысл монадического связывания}
      \begin{itemize}
          \item Что такое связывание (обычное)?
          \item Связывание значения с именем
      \end{itemize}
      \begin{columns}
        \begin{column}{0.5\textwidth}
\begin{minted}{haskell}
expr :: Double
expr =
  let a = 10
      b = 20
      c = 30
      d = 40
  in a + b * c / d
\end{minted}
        \end{column}
        \begin{column}{0.5\textwidth}
\begin{minted}{haskell}
expr' :: Double
expr' =
  (\a ->
   \b ->
   \c ->
   \d ->
     a + b * c / d)
  10 20 30 40
\end{minted}
        \end{column}
      \end{columns}
  \end{frame}
  \begin{frame}[fragile]{Глубокий смысл монадического связывания}
      \begin{itemize}
          \item А теперь через монаду Identity
          \item Она очень простая -- не делает \textit{ничего}
      \end{itemize}
      \begin{columns}
        \begin{column}{0.5\textwidth}
\begin{minted}{haskell}
expr' =
  (\a ->
   \b ->
   \c ->
   \d ->
     a + b * c / d)
  10 20 30 40
\end{minted}
        \end{column}
        \begin{column}{0.5\textwidth}
\begin{minted}{haskell}
expr'' =
  runIdentity $
    return 10 >>= \a ->
    return 20 >>= \b ->
    return 30 >>= \c ->
    return 40 >>= \d ->
      return a + b * c / d
\end{minted}
        \end{column}
      \end{columns}
  \end{frame}
  \begin{frame}[fragile]{Глубокий смысл монадического связывания}
      \begin{itemize}
          \item А теперь через do-нотацию
      \end{itemize}
      \begin{columns}
        \begin{column}{0.5\textwidth}
\begin{minted}{haskell}
expr'' =
  runIdentity $
    return 10 >>= \a ->
    return 20 >>= \b ->
    return 30 >>= \c ->
    return 40 >>= \d ->
      return a + b * c / d
\end{minted}
        \end{column}
        \begin{column}{0.5\textwidth}
\begin{minted}{haskell}
expr''' =
  runIdentity $ do
     a <- return 10
     b <- return 20
     c <- return 30
     d <- return 40
     return a + b * c / d
\end{minted}
        \end{column}
      \end{columns}
\end{frame}
  \section{Вычислительные эффекты}
\begin{frame}{Что такое вычислительный эффект}
    \begin{block}{Свойства чистых функций}
      \begin{itemize}
          \item \textbf{Полная определённость (тотальность)} -- даёт ответ для любого значения аргумента
          \item \textbf{Детерминированность} -- для заданных входных данных функция всегда возвращает одинаковый результат
          \item \textbf{Отсутствие побочных эффектов} -- не ссылаются на данные, значения которых могут измениться в процессе работы программы и не производит таких изменений
      \end{itemize}
    \end{block}
    \textbf{Вычислительный эффект} --- любое действие, нарушающее свойства чистых функций
\end{frame}
\begin{frame}{Какие бывают вычислительные эффекты}
    \begin{itemize}
        \item \textbf{Частичность} -- функция не определена для каких-либо значений аргумента (не возвращает значения для него, т.е. \textit{не завершается})
        \item \textbf{Недетерминированность}
          \begin{itemize}
              \item \textbf{Нестабильность} -- функция не всегда возвращает одинаковые значения для одного значения аргумента
              \item \textbf{Множественность} -- функция возвращает сразу несколько ответов
          \end{itemize}
        \item \textbf{Побочный эффект} -- функция не только возвращает результат, но и делает \textit{что-то ещё}
    \end{itemize}
\end{frame}
\section{Эффекты в Haskell}
\begin{frame}[fragile]{Эффект частичности}
\only<2>{\textbf{Maybe!}}
    \begin{columns}
      \begin{column}{0.5\textwidth}
\begin{minted}{haskell}
f :: Double -> Double
f d =
  let a = 10
      b = 20
      c = 30
  in a + b * c / d

>f 0
Infinity
\end{minted}
      \end{column}
      \begin{column}{0.5\textwidth}
      \pause
\begin{minted}{haskell}
f :: Double -> Maybe Double
f d =
  let a = 10
      b = 20
      c = 30
  in if (abs d < 0.0001)
     then Nothing
     else
       Just (a + b * c / d)
>f 0
Nothing
>f 1
Just 610.0
\end{minted}      
      \end{column}
    \end{columns}
\end{frame}
\begin{frame}[fragile]{Эффект частичности}
    \begin{columns}
      \begin{column}{0.5\textwidth}
\begin{minted}{haskell}
f :: Double -> Maybe Double
f d =
  let a = 10
      b = 20
      c = 30
  in if (abs d < 0.0001)
     then Nothing
     else
       Just (a + b * c / d)
\end{minted}
      \end{column}
      \begin{column}{0.5\textwidth}
\begin{minted}{haskell}
f :: Double -> Maybe Double
f d = do
  guard (abs d >= 0.0001)
  let a = 10
      b = 20
      c = 30
  return $ a + b * c / d
\end{minted}      
      \end{column}
    \end{columns}
\end{frame}
\begin{frame}[fragile]{\texttt{Maybe} как монада}
Реализуется достаточно просто:
\begin{minted}{haskell}
instance Monad Maybe where
  return = Just
  
  Nothing >>= _ = Nothing
  (Just x) >>= f = f x
  
  (Just _) >> a = a
  Nothing >> _ = Nothing
\end{minted}
\end{frame}
\begin{frame}{Эффект множественности}
    \begin{itemize}
        \item Квадратное уравнение может иметь два корня, один корень, не иметь корней
        \item \only<1>{\hs{solver :: Double -> Double -> Double -> ?}}
              \only<2>{\hs{solver :: Double -> Double -> Double -> [Double]}}
    \end{itemize}
\end{frame}
\begin{frame}[fragile]{Список как монада}
Определяется следующим образом:
\begin{minted}{haskell}
instance Monad [] where
  return x = [x]
  xs >>= k = concat (map k xs)
\end{minted}
\end{frame}
\begin{frame}[fragile]{Связь монады списка и list comprehension}
Следующие выражения эквивалентны:
    \begin{columns}
      \begin{column}{0.5\textwidth}
\begin{minted}{haskell}
triples =
  [(x,y,z) |
   x <- [1..10],
   y <- [1..10],
   z <- [1..10],
   x^2 + y^2 == z^2]
\end{minted}
      \end{column}
      \begin{column}{0.5\textwidth}
\begin{minted}{haskell}
triples = do
  x <- [1..10]
  y <- [1..10]
  z <- [1..10]
  guard $ x^2 + y^2 == z^2
  return (x, y, z)
\end{minted}
      \end{column}
    \end{columns}
\end{frame}
\begin{frame}{Ввод-вывод}
    \begin{itemize}
        \item Частичный (файл может отсутствовать, прочитали до EOF)
        \item Недетерминированный (файл может измениться)
        \item Имеет побочные эффекты (печать в файл)
    \end{itemize}
    \pause
    \begin{center}
        \textbf{Нечистый!} 
        
        \pause Но такой нужный...
    \end{center}
\end{frame}
\begin{frame}{Что почитать}
    \begin{itemize}
        \item \url{https://ruhaskell.org/posts/theory/2018/01/10/effects.html}
        \item \url{https://ruhaskell.org/posts/theory/2018/01/18/effects-haskell.html}
    \end{itemize}
\end{frame}
  \section{Q\&A}
\end{document}