\documentclass{beamer}
\usetheme{metropolis}
\usepackage[T2A]{fontenc}
\usepackage[utf8]{inputenc}
\usepackage[english,russian]{babel}
\usepackage{minted}
\usepackage{soul}

\newcommand{\hs}[1]{\mintinline{haskell}{#1}}
\newcommand{\bhs}[1]{\fbox{\mintinline{haskell}{#1}}}

\title{Декларативное программирование}
\subtitle{Семинар №13, группа 22215}
\author{Завьялов А.А.}
\date{28 ноября 2022 г.}
\institute{Кафедра систем информатики ФИТ НГУ}
\begin{document}
  \maketitle
\begin{frame}{Вычислительные эффекты}
   \begin{block}{Что такое}
   \vspace{2px}
      Если функция не просто переводит одно значение в другое, но \textit{делает что-то} ещё
      или \textit{по-особому} работает, говорят, что она содержит некоторый \textbf{вычислительный эффект}
   \end{block}
   \begin{block}{Какие бывают}
    \begin{itemize}
        \item Частичность \only<2->{-- \texttt{Maybe} \only<4->{(и не только)}}
        \item Недетерминированность
          \begin{itemize}
              \item \only<1-4>{Нестабильность}\only<5>{\alert{Нестабильность}}
              \item Множественность \only<3->{-- \only<3-4>{\texttt{[]}}\only<5>{\alert{\texttt{[]}}} \only<4->{(и не только)}}
          \end{itemize}
        \item \only<1-4>{Побочный эффект}\only<5>{\alert{Побочный эффект}}
    \end{itemize}
    \end{block}
\end{frame}
\begin{frame}{Монада списка и динамическое программирование}
    Определите функцию \hs{permutations :: [a] -> [[a]]}, которая возвращает все перестановки, которые можно получить из данного списка, в любом порядке.
    \pause
    \begin{block}{Перестановки разной степени сложности}
     \begin{itemize}
         \item перестановки пустого списка -- пустой список;
         \item перестановки одноэлементного списка -- тот же самый список;
         \item перестановки списка \hs{[x, y]} -- список \hs{[[x, y], [y,x]]};
         \item перестановки списка \hs{[x1, x2, ..., xn]} -- перестановки \hs{[x2, ..., xn]} с добавленным в начало \hs{x1}, а также перестановки \hs{[x1, x3, ..., xn]} с добавленным в начало \hs{x2} и т.д...
     \end{itemize}
    \end{block}
\end{frame}
\begin{frame}[fragile]{Сопоставление с образцом в монадах и \texttt{MonadFail}}
    \begin{columns}
    \begin{column}{0.5\textwidth}
\textbf{Переводчик на собачий язык}
    \begin{minted}{haskell}
dogefier :: 
  Maybe String -> Maybe String
dogefier (Just "hello") = 
  Just $ "henlo"
    \end{minted}
    
\textbf{Что выведет интерпретатор?}
\begin{minted}{haskell}
> dogefier $ return "hello"
\end{minted}
\begin{minted}{haskell}
> dogefier $ return "привет"
\end{minted}
    \end{column}
    \begin{column}{0.5\textwidth}
    \begin{figure}
        \centering
        \includegraphics[scale=0.3]{media/henlo.png}
    \end{figure}
    \end{column}
    \end{columns}
    \only<2>{\begin{center}
        \textbf{Попробуем исправить}
    \end{center}}
\end{frame}
\begin{frame}[fragile]{Метод \texttt{fail} и \texttt{do}-нотация}
    \begin{columns}
    \begin{column}{0.5\textwidth}
      \begin{block}{Выражения вида}
      
      \begin{minted}{haskell}
do { Just x1 <- action1
   ; x2      <- action2
   ; mk_action3 x1 x2 }
      \end{minted}
      \end{block}
    \end{column}
    \begin{column}{0.5\textwidth}
      \begin{block}{Превращаются в}
      
      \begin{minted}{haskell}
action1 >>= f
where f (Just x1) = 
  do { x2 <- action2
     ; mk_action3 x1 x2 }
      
      f _ = fail "..."
      \end{minted}
      \end{block}
    \end{column}
    \end{columns}
    \begin{itemize}
        \item Работает для всех монад, реализующих класс типов \texttt{MonadFail} (содержит метод \texttt{fail})
        \item \hs{fail :: Maybe a} $\equiv$ \hs{Nothing}
        \item \hs{fail :: [a]} $\equiv$ \hs{[]}
    \end{itemize}
\end{frame}
\begin{frame}[fragile]{Глубокий смысл \texttt{fail} для монады списка}
    \begin{columns}
    \begin{column}{0.4\textwidth}
\begin{block}{Выражение}
\begin{minted}{haskell}
evens =
 [x |
  x <- [1..],
  x `mod` 2 == 0]
\end{minted}
\end{block}
    \end{column}
    \begin{column}{0.6\textwidth}
\begin{block}{Эквивалентно следующему}
\begin{minted}{haskell}
evens = do
  x <- [1..]
  True <- return x `mod` 2 == 0
  return x
\end{minted}
\end{block}
    \end{column}
    \end{columns}
    \begin{itemize}
        \item Позволяет избавляться от ненужных значений;
        \item \texttt{guard} делает похожую работу, но:
          \begin{itemize}
              \item мы пока не знаем, как работает \texttt{guard};
              \item мы уже знаем, как работает паттерн-матчинг и \texttt{fail}.
          \end{itemize}
    \end{itemize}
\end{frame}
\begin{frame}[fragile]{Эффект нестабильности}
    Если ваша функция \hs{f :: a -> b} возвращает разные значения при одинаковом значении аргумента, значит она зависит от скрытого параметра.
    \pause\newline\newline
    Чистые функции не зависят от скрытых изменяющихся переменных. Нестабильность можно
    выразить, сделав зависимость явной:
\begin{minted}{haskell}
    f' :: a -> r -> b
\end{minted}
    \pause
    Можем дать \hs{e -> b} псевдоним:
\begin{minted}{haskell}
    type Reader r b = r -> b
    f' :: a -> Reader r b
\end{minted}
\end{frame}
\begin{frame}[fragile]{Монада \texttt{Reader}}
\begin{block}{Определение}
\begin{minted}{haskell}
newtype Reader r a = Reader { runReader :: r -> a }

instance Monad (Reader r) where
  return a = Reader $ \_ -> a
  -- Передаёт окружение в оба вычисления
  m >>= k = Reader $ \r -> let v = runReader m r
                           in runReader (k v) r
\end{minted}
\end{block}
\begin{block}{Полезные функции}
\begin{itemize}
    \item \hs{ask = Reader id :: Reader a a}
    \item \hs{asks f = Reader f :: (r -> a) -> Reader r a}
    \item \hs{local :: (r -> r) -> Reader r a -> Reader r a}
    
          \hs{local f m = Reader $ runReader m . f}
\end{itemize}
\end{block}
\end{frame}
\begin{frame}[fragile]{Побочный эффект}
    Если ваша функция \hs{f :: a -> b} не только возвращает значение, но и делает что-то ещё (например, изменяет глобальную переменную), значит, она имеет побочный эффект.
    \pause\newline\newline
    Чистые функции не могут сделать ничего, кроме подсчета единственного возвращаемого значения. Побочный эффект можно выразить, сделав побочный эффект частью результата функции:
\begin{minted}{haskell}
    f' :: a -> (b, s)
\end{minted}
\pause
Можем сделать \hs{(b,s)} осмысленным типом:
\begin{minted}{haskell}
    newtype Putter s b = Putter (b, s)
    f' :: a -> Putter s b
\end{minted}
\end{frame}
\begin{frame}[fragile]{Монада \texttt{Writer}}
\begin{block}{Определение}
\begin{minted}{haskell}
newtype Writer w a = Writer { runWriter :: (a, w) }

instance Monoid w => Monad (Writer w) where
  return x = Writer (x, mempty)
  Writer (x,v) >>= f = let (Writer (y, v')) = f x
                       in Writer (y, v `mappend` v')
\end{minted}
\end{block}
\begin{block}{Полезные функции}
\begin{itemize}
    \item \hs{tell :: Monoid w => w -> Writer w ()}
    \item \hs{execWriter :: Writer w a -> w}
\end{itemize}
\end{block}
\end{frame}
\begin{frame}{Что почитать}
    \begin{itemize}
        \item \url{https://ruhaskell.org/posts/theory/2018/01/10/effects.html}
        \item \url{https://ruhaskell.org/posts/theory/2018/01/18/effects-haskell.html}
    \end{itemize}
\end{frame}
  \section{Q\&A}
  \appendix
  \begin{frame}[fragile]{Монада \texttt{State}}
\begin{minted}{haskell}
newtype State s a = State { runState :: s -> (a,s) }

instance Monad (State s) where
  -- return :: a -> State s a
  return x = State $ \s -> (x, s)
  -- (>>=) :: State s a -> (a -> State s b) -> State s b
  State act >>= f = State $ \s ->
    let (a, s') = act s
    in runState (k a) s'
\end{minted}
\begin{block}{Полезные функции}
\begin{itemize}
    \item \hs{get       :: State s s}
    \item \hs{put       :: s -> State s ()}
    \item \hs{modify    :: (s -> s) -> State s ()}
\end{itemize}
\end{block}
  \end{frame}
\end{document}