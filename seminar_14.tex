\documentclass{beamer}

% Theme and customization
\usetheme{metropolis}
%\metroset{progressbar=foot}

% Define colors
\definecolor{magic}{RGB}{94, 81, 132}
\definecolor{darkmagic}{RGB}{82, 70, 115}

% Set theme colors
\setbeamercolor{normal text}{fg=magic}
\setbeamercolor{alerted text}{fg=darkmagic}

\usepackage[T2A]{fontenc}
\usepackage[utf8]{inputenc}
\usepackage[english,russian]{babel}
\usepackage{minted}
\usepackage{soul}

\newcommand{\hs}[1]{\mintinline{haskell}{#1}}
\newcommand{\bhs}[1]{\fbox{\mintinline{haskell}{#1}}}

\title{~~~~\includegraphics[scale=0.5]{media/harry-potter-title.pdf}}
\subtitle{~~~~~~~~~~~~~~~~~~~~~~~~~~и Функция-полукровка}
\author{Завьялов А.А.}
\date{5 декабря 2022 г.}
\institute{Кафедра систем информатики ФИТ НГУ}
\begin{document}
  \maketitle
  \begin{frame}[fragile]{Проблемы с вводом-выводом}
\begin{itemize}
    \item Все функции в Haskell -- чистые
    \item Хотим работать с вводом-выводом (потенциально "грязными операциями)
    \item \hs{getCharFromConsole :: Char} должна \textbf{всегда} возвращать одно и то же
    \pause
    \item \hs{getCharFromConsole :: RealWorld -> (RealWorld, Char)} всегда даёт одинаковый
    ответ при одинаковом состоянии \hs{RealWorld}
    \item Окружающий мир непрерывно изменяет свое состояние
    \item Невозможно или тяжело отследить и локализовать изменения
    \item Нужно ограничивать доступ к \hs{RealWorld}
    \pause
    \item \hs{getChar :: IO Char}. \pause \hs{IO}?
\end{itemize}
  \end{frame}
  \begin{frame}{Что такое \texttt{IO}?}
      \begin{itemize}
          \item Тип \hs{IO a} говорит нам -- "во время вычисления значения типа \hs{a} может случиться операция ввода-вывода"
          \item \hs{IO} приблизительно определяется как:
                \hs{newtype IO a = IO (RealWorld -> (RealWorld, a))}
           \item Документация говорит: "RealWorld is \textbf{deeply magical}"
           \item Значениями \hs{RealWorld} нельзя манипулировать, он существует только
           на уровне системы типов
           \pause
           \item Говорят, что выражения с \hs{IO a} это не функции\pause, а \textbf{действия ввода вывода} (IO actions)
      \end{itemize}
  \end{frame}
  \begin{frame}[fragile]{Монада \texttt{IO}}
    \hs{IO} можно примерно определить как

\begin{minted}{haskell}
instance Monad IO where
  return x = IO $ \w -> (w, x)
  m >>= k = IO $ \w ->
    case m w of
      (w', a) -> k a w'
\end{minted}

\begin{block}{Что важно}
    \begin{itemize}
        \item Нельзя достать значения из \hs{IO}
        \item Побочный эффект (операция ввода-вывода) действия происходит один раз
        \item Побочные эффекты происходят в заданном порядке
    \end{itemize}
\end{block}
  \end{frame}
\begin{frame}{Основные функции консольного ввода-вывода}
    \begin{itemize}
        \item Ввод
        \begin{itemize}
            \item \hs{getChar :: IO Char}
            \item \hs{getLine :: IO String}
            \item \hs{getContents :: IO String}
        \end{itemize}
    \item Вывод
    \begin{itemize}
        \item \hs{putChar :: Char -> IO ()}
        \item \hs{putStr, putStrLn :: String -> IO ()}
        \item \hs{print :: Show a => a -> IO ()}
    \end{itemize}
    \item Ввод-вывод
    \begin{itemize}
        \item \hs{interact :: (String -> String) -> IO ()}
    \end{itemize}
    \end{itemize}
\end{frame}
\begin{frame}{Модуль \texttt{System.Random}}
    \begin{itemize}
        \item Вообще-то, генераторы псевдослучайных чисел не зависят от \textit{окружающего мира}
        \pause\item Но могут...
        \pause\item Это делает \textit{некоторые} генераторы действиями ввода-вывода
    \end{itemize}
    \pause
    \begin{block}{Установка \texttt{System.Random}}
    \vspace{5px}
        \mintinline{bash}{$ cabal install --lib random}
    \end{block}
    \pause\begin{block}{Полезные функции}
        \begin{itemize}
            \item \hs{randomIO :: (Random a, MonadIO m) => m a}
            \item \hs{randomRIO :: (Random a,MonadIO m)=>(a,a)->m a}
        \end{itemize}
    \end{block}
\end{frame}
\begin{frame}{Полезные функции для работы с монадами}
    Эти функции определены для \textbf{любых} монад, не только \hs{IO}

    \begin{itemize}
    \item \hs{mapM :: (Traversable t, Monad m) =>} \hs{(a -> m b) -> t a -> m (t b)}
    \item \hs{forM :: (Traversable t, Monad m) =>} \hs{t a -> (a -> m b) -> m (t b)}
    \item \hs{sequence :: (Traversable t, Monad m)} \hs{=> t (m a) -> m (t a)}
    \end{itemize}
    Есть версии, возвращающие \hs{m ()}:  \texttt{mapM\_, forM\_, sequence\_}
  \end{frame}
  \begin{frame}{Тёмные искусства -- изменяемые ссылки}
      \begin{itemize}
          \item \textbf{Не повторяйте этого дома}
          \item \hs{import Data.IORef (newIORef,readIORef,writeIORef)}
      \end{itemize}
  \end{frame}
  \begin{frame}{Темнейшее волшебство}
      \begin{itemize}
          \item \textbf{Не стоит вам этим пользоваться. \underline{Никогда.}}
          \pause\item \hs{import System.IO.Unsafe}
      \end{itemize}
  \end{frame}
  \section{Q\&A}
\end{document}