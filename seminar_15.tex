\documentclass{beamer}
\usetheme{metropolis}
\usepackage[T2A]{fontenc}
\usepackage[utf8]{inputenc}
\usepackage[english,russian]{babel}
\usepackage{minted}
\usepackage{soul}

\newcommand{\hs}[1]{\mintinline{haskell}{#1}}
\newcommand{\bhs}[1]{\fbox{\mintinline{haskell}{#1}}}

\title{Декларативное программирование}
\subtitle{Семинар №15, группа 22215}
\author{Завьялов А.А.}
\date{12 декабря 2022 г.}
\institute{Кафедра систем информатики ФИТ НГУ}
\begin{document}
  \maketitle
\begin{frame}[fragile]{Обработка ошибок в Haskell}
\begin{block}{Что произойдёт, если ввести ''hello''?}
\begin{minted}{haskell}
main = do
  x <- getLine
  True <- return $ length x == 10
  putStrLn x
\end{minted}
\end{block}
\pause
\begin{block}{Произойдёт}    
\begin{minted}{haskell}
*** Exception: user error
    (Pattern match failure in do expression)
\end{minted}
\end{block}
\pause
\begin{center}
\textbf{Q: Почему?} \pause
\\
A: Давайте обратимся к документации
\end{center}
\end{frame}
\begin{frame}{Исключения в Haskell}
    Модуль \hs{Control.Exception}
    \begin{block}{Полезные функции}
        \begin{itemize}
            \item \hs{catch :: Exception e=>IO a->(e->IO a)->IO a}
            \item \hs{catches :: IO a -> [Handler a] -> IO a}
            \item \hs{try :: Exception e => IO a -> IO (Either e a)}
            \item \hs{evaluate :: a -> IO a}
            \item \hs{throwIO :: Exception e => e -> IO a}
            \item \hs{handle}, \hs{handles}, ...
        \end{itemize}
    \end{block}
\end{frame}
\begin{frame}[fragile]{Тип данных \texttt{Either}}
\begin{minted}{haskell}
data Either a b
  = Left a
  | Right b
\end{minted}
    \begin{itemize}
        \item ''Левое значение'' -- ошибка
        \item ''Правое значение'' -- результат
        \item \textit{Реализует класс типов \texttt{Monad}}
    \end{itemize}
    \begin{block}{Полезные функции}
        \begin{itemize}
            \item \hs{either :: (a->c)->(b->c) -> Either a b -> c}
            \item \hs{lefts :: [Either a b] -> [a]}, \hs{rights}
            \item \hs{fromRight :: b -> Either a b -> b}, \hs{fromLeft}
        \end{itemize}
    \end{block}
\end{frame}
\begin{frame}{Тестирование программ на Haskell}
    \begin{block}{Виды тестирования}
        \begin{itemize}
            \item \pause Ручное
            \item \pause Автоматизированное
            \begin{itemize}
                \item \pause Юнит-тестирование
                \item \pause Проверка свойств (property-based testing)
            \end{itemize}
        \end{itemize}
    \end{block}
    \pause
    \begin{block}{Фреймворки для тестирования}
        \begin{itemize}
            \item HUnit
            \item QuickCheck
        \end{itemize}
    \end{block}
\end{frame}
\section{Q\&A}
\end{document}