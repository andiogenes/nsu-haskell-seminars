\documentclass{beamer}
\usetheme{metropolis}
\usepackage[T2A]{fontenc}
\usepackage[utf8]{inputenc}
\usepackage[english,russian]{babel}
\usepackage{minted}
\usepackage{soul}

\newcommand{\hs}[1]{\mintinline{haskell}{#1}}
\newcommand{\bhs}[1]{\fbox{\mintinline{haskell}{#1}}}

\title{Декларативное программирование}
\subtitle{Семинар №16, группа 22215}
\author{Завьялов А.А.}
\date{19 декабря 2022 г.}
\institute{Кафедра систем информатики ФИТ НГУ}
\begin{document}
  \maketitle
\section{А что в других языках?}
\begin{frame}{Чистые функции}
    \begin{itemize}
        \item \only<1>{Детерминированные}\only<2->{\textbf{Детерминированные}}
        \item \only<1>{Без побочных эффектов}\only<2->{\textbf{Без побочных эффектов}}
        \item \pause Гарантии!
        \item \pause Можно написать на любом современном языке
    \end{itemize}
\end{frame}
\begin{frame}{Функции}
    \begin{block}{Функции высшего порядка}
    \begin{itemize}
        \item Может принимать функцию в качестве аргумента
        \item Или возвращать функцию в качестве результата
        \item Пример: \hs{map :: (a -> b) -> [a] -> [b]}
    \end{itemize}
    \end{block}
    \begin{block}{Функции как объекты первого класса}
        \begin{itemize}
            \item \textit{Можно передавать в другие функции}
            \item \textit{Возвращать в качестве результата}
        \end{itemize}
    \end{block}
    \begin{block}{Где встречается}
        \begin{itemize}
            \item В языках с поддержкой анонимных функций
            \item В Cи? \textbf{Нет!}
            \item В С++? \textbf{Да!} (А также в Java, C\#, Python и т.д.)
        \end{itemize}
    \end{block}
\end{frame}
\begin{frame}{Композиция функций}
    \begin{block}{Пример использования}
        \begin{itemize}
            \item Обработка коллекций
        \end{itemize}
    \end{block}
    \begin{block}{Где встречается}
        \begin{itemize}
            \item Java (Stream API, since Java 8), Scala, Kotlin
            \item C\# (LINQ)
            \item JavaScript
            \item \textbf{Unix Pipes}! (Bash и другие оболочки)
            \item F\#, Elixir, Clojure, Elm и другие функциональные языки
            \item C++ (\texttt{std::ranges}, since C++ 20)
        \end{itemize}
    \end{block}
\end{frame}
\begin{frame}{Ленивые вычисления}
    \begin{block}{Где встречается}
        \begin{itemize}
            \item Scheme, Racket
            \item Scala
            \item Clojure
            \item Отчасти в Java, C\# и C++ в рамках Stream API, LINQ и \texttt{std::ranges}
        \end{itemize}
    \end{block}
    \begin{block}{Можно реализовать и самому}
        \begin{itemize}
            \item Как-то создать отложенные вычисления
            \item Уметь запускать вычисления, когда они понадобятся
            \item На самом деле много тонкостей (Haskell их учитывает)
        \end{itemize}
    \end{block}
\end{frame}
\begin{frame}{Алгебраические типы данных и сопоставление с образцом}
    \begin{block}{Пример использования}
        \begin{itemize}
            \item Моделирование предметной области
            \item Обработка сложных структур данных
        \end{itemize}
    \end{block}
    \begin{block}{Где встречается}
        \begin{itemize}
            \item Языки семейства ML: Standard ML, OCaml (и F\#)
            \item Rust
            \item Scala
            \item Kotlin
            \item \textbf{Pattern-matching идёт в мейнстрим}:
            \begin{itemize}
                \item C\# 7.0
                \item Java 16
                \item Python 3.10 (!)
            \end{itemize}
        \end{itemize}
    \end{block}
\end{frame}
\begin{frame}{Функторы}
    \begin{block}{Пример использования}
        \begin{itemize}
            \item Обработка ситуаций, когда значения может не быть
        \end{itemize}
    \end{block}
    \begin{block}{Где встречается}
        \begin{itemize}
            \item Kotlin
            \item C\#
            \item JavaScript
            \item Scala
        \end{itemize}
    \end{block}
\end{frame}
\begin{frame}{Классы типов}
    \begin{block}{Пример использования}
        \begin{itemize}
            \item Написание \textit{обобщённого} кода
        \end{itemize}
    \end{block}
    \begin{block}{Где встречается}
        \begin{itemize}
            \item Scala
            \item Rust
            \item C++ 20 (concepts)
        \end{itemize}
    \end{block}
\end{frame}
\begin{frame}{Монады и их друзья}
    \begin{block}{Пример использования}
        \begin{itemize}
            \item Моделирование последовательности вычислений
            \item Моделирование вычислительных эффектов
            \item Ввод-вывод
            \item И не только...
        \end{itemize}
    \end{block}
    \begin{block}{Где встречается}
        \begin{itemize}
            \item Scala (Cats, Cats Effect, ZIO, Scalaz)
            \item Kotlin (Arrow)
            \item PureScript, OCaml и другие функциональные языки
        \end{itemize}
    \end{block}
\end{frame}
\section{Q\&A}
\end{document}