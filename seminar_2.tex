\documentclass{beamer}
\usetheme{metropolis}           % Use metropolis theme
\metroset{numbering=fraction}
\usepackage[T2A]{fontenc}
\usepackage[utf8]{inputenc}
\usepackage[english,russian]{babel}
\title{Декларативное программирование}
\subtitle{Семинар №2, группа 22215}
\author{Завьялов А.А.}
\date{12 сентября 2022 г.}
\institute{Кафедра систем информатики ФИТ НГУ}
\begin{document}
  \maketitle
  \addtocounter{framenumber}{1}
  \begin{frame}{Гугл Классная Комната}
  \includegraphics[width=4cm]{media/classroom-invite-22215}
  
  \url{https://bit.ly/hask22215classroom}
  \end{frame}
  \begin{frame}{Домашняя работа №1}
      \begin{itemize}
          \item Написать функцию, вычисляющую сумму ряда геометрической прогрессии до $n$-го члена (по определению)
          \item Написать функцию, вычисляющую сумму ряда геометрической прогрессии с заданной точностью $\varepsilon$
      \end{itemize}
  \end{frame}
  \begin{frame}{Попытка №2 --- Что почитать?}
      \begin{itemize}
          \item Изучай Haskell во имя добра! -- Миран Липовача
          \begin{itemize}
              \item Бесплатно, на английском: \url{http://learnyouahaskell.com/}
          \end{itemize}
          \item Программируй на Haskell -- Уилл Курт
      \end{itemize}
      \begin{itemize}
          \item \alert{\href{https://hbr.github.io/Lambda-Calculus/lambda2/lambda.html}{Programming with Lambda Calculus}} -- Helmut Brandl
      \end{itemize}
  \end{frame}
  \section{Q\&A}
\end{document}
