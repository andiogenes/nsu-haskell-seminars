\documentclass{beamer}
\usetheme{metropolis}           % Use metropolis theme
\metroset{numbering=fraction}
\usepackage[T2A]{fontenc}
\usepackage[utf8]{inputenc}
\usepackage[english,russian]{babel}
\title{Декларативное программирование}
\subtitle{Семинар №3, группа 22215}
\author{Завьялов А.А.}
\date{19 сентября 2022 г.}
\institute{Кафедра систем информатики ФИТ НГУ}
\begin{document}
  \maketitle
  \addtocounter{framenumber}{1}
  \begin{frame}{Как работают вызовы функций (в императивных языках)}
    \begin{itemize}
        \item \textit{Где-то} в памяти расположен \textit{стек вызовов}
        \item При вызове функции: \begin{enumerate}
            \item Вычисляются ее параметры
            \item Значения параметров записываются в стек
            \item В стек помещается адрес возврата
            \item Управление передается исполняемой функции
            \item Функция резервирует место под свои локальные переменные и, \textit{возможно}, возвращаемое значение
            \item Функция работает
            \item В случае \textit{return}:\begin{enumerate}
                \item функция освобождает свой стековый кадр
                \item Передает управление по адресу возврата
            \end{enumerate} 
        \end{enumerate}
    \end{itemize}
  \end{frame}
  \begin{frame}{Стек вызовов и рекурсия}
      \begin{itemize}
          \item Стек имеет конечный размер (обычно 1MB, 8 MB)
          \item Рекурсивно можно задать незавершимые вычисления
          \item Каждый рекурсивный вызов приводит к росту стека
          \item ?
          \item \pause\alert{Stack Overflow}
      \end{itemize}
  \end{frame}
  \begin{frame}{Tail recursion to the rescue}
    \begin{block}{Хвостовая рекурсия}
        \begin{itemize}
            \item Рекурсивный вызов -- последняя операция перед выходом из функции
            \item Последняя операция $\rightarrow$ параметры и локальные переменные уже не используются
            \item Можем заменить значения на стеке и передать управление функции без создания нового стекового кадра
            \item Заменили рекурсию на \textit{GOTO}!
        \end{itemize}
    \end{block}
  \end{frame}
    \begin{frame}{Рекурсивные функции}
    \begin{block}{Факториал}
      $$ n! = \begin{cases}
			1, & n = 0\\
            n * (n - 1)!, & \text{иначе}
		 \end{cases} $$
	 \end{block}

    \begin{block}{Числа Фибоначчи}
        $$ F_0 = 1,~F_1 = 1, $$
        $$F_n = F_n-1 + F_n-2, n \ge 2$$
    \end{block}
  \end{frame}
  \begin{frame}{Домашняя работа №2}
      \includegraphics[width=12cm]{media/wbrb}
  \end{frame}
  \section{Q\&A}
\end{document}
