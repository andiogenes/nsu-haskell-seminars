\documentclass{beamer}
\usetheme{metropolis}           % Use metropolis theme
\metroset{numbering=fraction}
\usepackage[T2A]{fontenc}
\usepackage[utf8]{inputenc}
\usepackage[english,russian]{babel}

\title{Декларативное программирование}
\subtitle{Семинар №7, группа 22215}
\author{Завьялов А.А.}
\date{17 октября 2022 г.}
\institute{Кафедра систем информатики ФИТ НГУ}
\begin{document}
  \maketitle
  \begin{frame}{Реклама}
    \begin{figure}
        \centering
        \includegraphics[width=0.95\textwidth]{media/tail-recursion.jpg}
    \end{figure}
    \begin{center}\url{https://www.youtube.com/watch?v=Slgjzt5Zado}\end{center}
  \end{frame}
  \section{Алгебраические типы данных}
  \section{Конструирование типов с помощью newtype}
  \section{Полиморфмизм, вывод типов и все-все-все}
  \begin{frame}{Полиморфизм, вывод типов и все-все-все}
      \begin{block}{Что почитать?}
      \begin{itemize}
          \item Введение в теорию языков программирования --- Ж.~Довек, Ж-Ж. Леви
          \item Типы в языках программирования --- Бенджамин Пирс (TaPL)
      \end{itemize}
      \end{block}
  \end{frame}
  \section{Q\&A}
\end{document}